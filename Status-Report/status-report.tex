\documentclass{article}

% Margins - see http://mirror.its.dal.ca/ctan/macros/latex/contrib/geometry/geometry.pdf for more info
\usepackage[left=3cm,top=3cm,right=3cm,bottom=3cm]{geometry}
\usepackage{graphicx}

% Temporary bugfix for BibTeX
% Found at http://tex.stackexchange.com/a/311428
\makeatletter
\def\blx@maxline{77}
\makeatother

\bibliography{proposal}

\title{SYSC4907 Sensor Based Access Control Project Update}
\author{
	Craig Shorrocks \\
	100887781
	\and
	Jessica Morris \\
	100882290
	\and
	Richard Perryman \\
	100887250
}

\date{\today}

\begin{document}

\maketitle

\begin{center}
Supervisors: Shikharesh Majumdar and Chung-Horng Lung
\end{center}

\pagebreak

\section{Abstract}

This report outlines the work that has been done so far on the Sensor Based Access Control System (SBACS). There are
three main components to the system: the hardware and software that manages the lock units, the server which stores
information about the system, and the software that handles user interations with the server and lock units.

At this time, most of the work on the lock unit technology as well as the server has been completed. The majority of the
remaining work to be done is on the access points, in particular, a web application to run administrative or user tasks
has only been planned.

Aside from this missing element, we are mostly on track compared to our schedule from our initial proposal. We are in
the middle of preliminary integration testing, and intend to have the system fully functional by the end of December.
From there, we can work on refining the design of the system, testing the system, and handling the other course
requirements like the rinal report.

\section{Hardware}

\section{Server}

\section{Access Points}

There were meant to be two main access points to the SBACS system: a phone application as well as a web portal. Both
systems allow users to manage their various identities and the locks that they associate them with. Both also expose
administrative capabilities for service providers. The phone application also had to provide a way for NFC to be used
with locks associated with the user.

The phone application currently does not authenticate users, but otherwise can perform all of the required actions.
The authentication implementation needs to be worked on in conjunction with the server adding the notion of sessions or
something similar. The web application currently doesn't exist, but should largely mimic the phone application in
design.

\section{Conclusion}

\end{document}
