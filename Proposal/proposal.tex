\documentclass{article}

\usepackage[backend=bibtex,style=numeric]{biblatex}
\usepackage{lipsum}

\bibliography{proposal}

\title{Fourth Year Project Proposal}
\author{
	Craig Shorrocks \\
	100887781
	\and
	Jessica Morris \\
	100882290
	\and
	Richard Perryman \\
	100887250
}

% Might be better to use the last date we edit this on,
% but for now I was too lazy to keep updating this
\date{\today} 

\begin{document}

\maketitle

\begin{center}
% Remember to ask if we should
% add that other professor (?)
Supervisor: Shikharesh Majumdar 
\end{center}

\pagebreak

\section{Objective}

As technology becomes more and more prevalent in our lives, our identities become more and more intertwined with the technologies that we use daily. This melding has become extremely prevalent in some areas. Some such areas are professional networking with LinkedIn, or banking with the advent of online account management. The vast majority of banking interactions are done over the internet \textless citation needed\textgreater. This popularity may be derived from how convenient and secure handling money online is. However, certain aspects of our day to day lives haven't yet been graced by the benefits of electronic security and automation.

Physical locks and keys are still widely used for several tasks where electronic locks could be used instead. House locks, bicycle locks, and locker locks are frequently physical locks. Changing to electronic locks could help streamline all of these locks, reducing the number of keys to remember and improving security by reducing the likelihood that the key could be faked by an attacker. Some applications for this have already been found: for example, Walmart has a system where customers can order products to be placed in lockers with electronic keys called Grab-and-Go \cite{WALMART}. Such a system greatly reduces the work involved in getting a key (in this case, a PIN instead of a physical key or combination) to the customer and increases the security of the lockers by reducing the number of points of failure.

This proposal outlines a system that will expand upon such a concept to further tie security and identity to the electronics we use most: our phones. Using technologies like near-field communication (NFC) sensors and quick response (QR) codes, the identity associated with a phone can be used as identification for anything. This represents a huge advantage with respect to convenience, and it would even further lower the number of possible failure points in security.

\section{Background}

\lipsum[1]

\section{Procedure} %This title feels bad (?)

\lipsum[1]

\section{Methods}

\lipsum[1]

\section{Time Table}

\lipsum[1]

\section{Components}

\lipsum[1]

\pagebreak

\printbibliography

\end{document}